\typeout{ ====================================================================}
\typeout{ this is file main.tex, created at 20-Mar-2015               }
\typeout{ maintained by Gustavo Rabello dos Anjos                             }
\typeout{ e-mail: gustavo.rabello@gmail.com                                   }
\typeout{ ====================================================================}

\documentclass[a4paper,portuguese,12pt]{article}

\usepackage{setspace}
\usepackage[hang,footnotesize]{caption}
\usepackage[utf8]{inputenc}
\usepackage{graphicx}
\usepackage[colorlinks=true]{hyperref}
\hypersetup{
  linkcolor=black,
  citecolor=blue,
}
\usepackage{semtrans}
\usepackage[margin=2.5cm]{geometry}
\usepackage[portuguese]{babel}
\usepackage[fixlanguage]{babelbib}
\selectbiblanguage{portuguese}
\usepackage[svgnames]{xcolor} % Specify colors by their 'svgnames'; list of all colors available: http://www.latextemplates.com/svgnames-colors
\usepackage{titlesec}
\usepackage[numbers]{natbib}
\usepackage{ifthen}
\usepackage{fancyhdr}
\columnsep=8mm

\setlength{\parskip}{2mm}
\setlength{\parindent}{0mm} % Default is 15pt.



%%%%%%%%%%%%%%%%%%%%%%%%%%%%%%%%%%%%%%%%%%%%%%%%%%%%%%%%%%%%%%%%%%%%%%%%%%%%%%%%%%
\begin{document}

\thispagestyle{empty}

\begin{minipage}{0.72\linewidth}
	\normalsize\textbf{UNIVERSIDADE DO ESTADO DO RIO DE JANEIRO\\
	              FACULDADE DE ENGENHARIA\\
				  DEPARTAMENTO DE ENGENHARIA MECÂNICA}
\end{minipage}
\begin{minipage}{0.27\linewidth}
	\flushright
	\includegraphics[height=16mm]{figs/logouerj.jpg}
	\hspace{.5cm}
	\includegraphics[height=16mm]{figs/fen.png}
\end{minipage}

\hrulefill

\Large \color{NavyBlue} \textbf{Monitoria: Dinâmica dos Fluidos}\\
\color{Black}\\ 
\normalsize \texttt{Orientador: Prof. Norberto Mangiavacchi}\\
\normalsize \texttt{e-mail: norberto.mangiavacchi@gmail.com}\\
\normalsize \texttt{Data de divulgação: \today}

\vspace{1cm}

\underline{INSCRIÇÃO:}

Poderão inscrever-se os alunos matriculados no(s) curso(s) de ENGENHARIA
MECÂNICA da UERJ e que atendam aos seguintes requisitos:

\begin{itemize}
	\item Estarem aprovados na disciplina objeto da monitoria;
	\item Não terem sofrido sanção disciplinar de suspensão há menos de
	1 (um) ano;
	\item Aluno(a) participante de qualquer estágio ou bolsa poderá se
	inscrever mas, se aprovado, deverá pedir desligamento, caso queira
	assumir a monitoria.
\end{itemize}

\underline{SELEÇÃO:}
\begin{itemize}
	\item O processo de seleção será realizado no dia 27.04.2015, às
	11h no seguinte local: Campus Fonseca Teles, prédio anexo, Laboratório
	GESAR;
	\item Processo de seleção para preenchimento de 01 vaga;
	\item Tipo de avaliação: entrevista, coeficiente de rendimento (CR)
	e currículo;
	\item Programa: Equações básicas diferenciais: continuidade,
	quantidade de movimento. Análise dimensional e semelhança.
	Escoamento rotacional e irrotacional. Escoamento incompreensível
	viscoso interno e externo. Escoamento desenvolvido. Camada limite.
	Escoamento sobre corpos imersos. Arraste e sustentação. Perfil
	aerodinâmico. Escoamento compressível. Velocidade do som. Condições
	de referência: estagnação e crítica. Escoamento isoentrópico em
	bocais e difusores. Escoamento de dutos de áarea constante:
	escoamento de Fanno e Rayleigh. Choque normal.
\end{itemize}

\underline{BIBLIOGRAFIA:}
\begin{itemize}
	\item Robert W. Fox, Alan T. McDonald, Philip J., Introdução à
	Mecânica dos Fluidos -- LTC;
	\item Panton, R.L. Incompressible Fluid Flow -- Wiley;
	\item José Pontes,  Norberto Mangiavacchi, Fenômenos de
	Transferência - Com Aplicações às Ciências Físicas e à Engenharia.
	Volume 1: Fundamentos -- apostila;
\end{itemize}

\end{document}








\typeout{ ****************** End of file main.tex ****************** }

